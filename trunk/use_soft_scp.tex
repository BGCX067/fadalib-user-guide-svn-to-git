\paragraph{Static Control Programs}
\begin{figure}[!h]
\begin{footnotesize}
\begin{lstlisting}
void    MatrixMultiply(int A[N][L], int B[L][M], int T[N][M]){
int i;
int j;
int k;
for(i=0;i<N;i++)
   for(j=0;j<M;j++){
        T[i][j]=0;
        for(k=0;k<L;k++)
                T[i][j]+=A[i][k]*B[k][j];
        }
\end{lstlisting}
\end{footnotesize}
\caption{\texttt{MatMul.c}: Static Control Program Example}
\label{pgm:matmul}
\end{figure}




\begin{figure}[!h]
\begin{scriptsize}
\begin{verbatim}
 $ fadatool -i MatMul.c
/////////////////////////////////////////////////
Symbolic constants are  = ( N, M, L )
/////////////////////////////////////////////////
********************************************
ID		:0
Counters	:(  )
Domain		:true
Written variable:
Read variable:
********************************************
ID		:1
Counters	:( i )
Domain		:((i >= 0) and (i <= N-1))
Written variable:
Read variable:
********************************************
ID		:2
Counters	:( i, j )
Domain		:(((i >= 0) and (i <= N-1))) and ((j >= 0) and (j <= M-1))
Written variable:   T[ i ][ j ]
Read variable:
********************************************
ID		:3
Counters	:( i, j )
Domain		:(((i >= 0) and (i <= N-1))) and ((j >= 0) and (j <= M-1))
Written variable:
Read variable:
********************************************
ID		:4
Counters	:( i, j, k )
Domain		:((((i >= 0) and (i <= N-1))) and ((j >= 0) 
                                  and (j <= M-1))) and ((k >= 0) and (k <= L-1))
Written variable:   T[ i ][ j ]
Read variable:
   T[ i ][ j ]

         if ( k >= 1 )
            < 4 : i, j, k-1 > 
         else
            < 2 : i, j >

   A[ i ][ k ]
      _|_

   B[ k ][ j ]
      _|_

/////////////////////////////////////////////////
================ Parallel Loops Detector:
      Parallel loops are : ( i, j ) 
================================
\end{verbatim}
\end{scriptsize}
\caption{Console Output for Dataflow Analysis of \texttt{MatMul.c} (Figure~\ref{pgm:matmul})}
\label{output:matmul}
\end{figure}

\verb|FADAtool| starts by detecting the symbolic constants of the program (first line in Figure~\ref{output:matmul}). Then it computes the guarded references from the statements of the original program. In fact, \verb|FADAtool| does not need the AST of the program. The required information are:
\begin{itemize}
 \item Which variables are read and written,
 \item within which loops (defined by their counters),
 \item and, what is the condition for which these variables can be read and written.
\end{itemize}

For example, the second assignment (\verb|ID#4|) reads three references (\verb|T[i][j]|, \verb|A[i][k]|, and \verb|B[k][j]|), one is written (\verb|T[i][j]|). These references are enclosed by loops \verb|i, j, k|, and guarded by the domain condition: ($ 0 \leq i \leq N-1,  0 \leq j \leq M-1, 0 \leq k \leq L-1$).

For each read variable, \verb|FADAtool| computes its source and expresses it in a quast. So, the definition of \verb|T[i][j]| read by \verb|ID#4| during iteration \verb|(i, j, k)| is (also given in Figure~\ref{output:matmul}):


\begin{verbatim}
         if ( k >= 1 )
            < 4 : i, j, k-1 > 
         else
            < 2 : i, j >
\end{verbatim}

And the source of \verb|A[i][j]| (read by \verb|ID#4| during the iteration \verb|(i, j, k)|) is :
\begin{verbatim}
      _|_
\end{verbatim}
In the console output, $\perp$ or Bottom  means that the source is undefined (out of analysedyzed scope). \verb|<ID : ITERATION_VECTOR >|  means that the value was produced by the statement \verb|ID|, during the iteration \verb|ITERATION_VECTOR|.


The exact source are computed by \verb|FADAtool| for \verb|T[i][j]|. It is presented as a quast, and should be read as : if \verb|A[i][j]| is read during the first iteration of the \verb|k|-loop, so its value is produced by assignment \verb|ID#2|. Otherwise, it its value is produced by assignment \verb|ID#4| executed during the previous iteration of the \verb|k|-loop, but int the same \verb|i-j|-loop iteration.


The third part of the output plots the parallel loops. Out parallel loops detector is elementary, it considers loops which do not carry dependences. This part is under constant improvments.
After a successful installation, FADAtool can be invocated this way :

\begin{verbatim}
 $ fada [options]
\end{verbatim}

\label{adaan_options}

Where options are :

\begin{itemize}
 \item \verb|--input=file| or \verb|-i file|    : specify the input file.
 \item Options relative to the FADA processing :
 \begin{itemize}
	\item \verb|--structural-analysis| or \verb|-s| :  apply the structural analysis.


 \end{itemize}
 \item Output and printing options:
    \begin{itemize}
        \item \verb|--dir=dir| or \verb|-d dir|        : set \verb|dir| as working and output directory (\verb|./| is taken as default value). if does not exist it will be created.
	\item \verb|--format=FORMAT| or \verb|-f FORMAT|, where \verb|FORMAT| in :
	\begin{itemize}
		\item \verb|html+tex| : print definitions in browsable html pages.
		\item \verb|html+dot| : print definitions in browsable html pages.
		\item \verb|console| : print definitions in browsable html pages.
	\end{itemize}

        \item \verb|--tree| or \verb|-T|        : print definitions in Quasi Affine Selection trees.
        \item \verb|--bottom| or \verb|-b|        : consider a node "bottom" for not initialized variables in the dataflow graph.
        \item \verb|--log| or\verb|-L|               : write results in a file.
        \item \verb|--fuzzy| or \verb|-f|            : remove constraints involving new parameters.
        \item \verb|--dataflowgraph| or \verb|-g|    : generate a dataflow graph.
        \item \verb|--verbose| or \verb|-v|          : print extra information for debugging.
        \item \verb|--help| or \verb|-h|        : print a help screen and exit.
        \item \verb|--version| or \verb|-V|          : print the version and exit.
     \end{itemize}
\end{itemize}

Examples :
\begin{verbatim}
  $ fada -i input.c 
\end{verbatim}
Perform dataflow analysis, results are printed in the standard output file.\\
\begin{verbatim}
$ fada -i input.c -H -d dir 
\end{verbatim}
Results are provided in html pages created in the directory \verb|dir|.\\

\begin{verbatim}
 $ fada -i input.c -g -d dir
\end{verbatim}
Compute the dataflow graph (automatically converted into a png picture), the file is created in the directory \verb|dir|.\\
\begin{verbatim}
   $ fada -i input.c -cg
\end{verbatim}
Show (on edges) the condition of validity of dependence.\\


\subsection{Input file}

The input file have to be a program or a function written in a subset of the C-language. Some restrictions are to be taken in account :
\begin{itemize}
 \item No use of \verb|struct|, structures have to be converted into arrays. For example : \verb|s.tab[i]| can be converted into \verb|s[offset_tab][i]| where \verb|offset_tab| is the offset of the field \verb|tab| in the structure \verb|s|.
 \item \verb|for| loops are only those who express an \textbf{upper} and \textbf{lower bounds} for\textbf{ one single loop counter}. Otherwise, they have to be converted to \verb|while| loops. Currently, the only supported stride is 1.
\end{itemize}
